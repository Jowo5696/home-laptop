\documentclass{article}

\usepackage[paperwidth=1920px,paperheight=1080px]{geometry}
\usepackage[T1]{fontenc}
\usepackage{amsmath,amssymb}
\usepackage[ISO]{diffcoeff}
\usepackage{xcolor}
\usepackage{color}

\definecolor{bg}{RGB}{69,0,128}
\definecolor{fg}{RGB}{129,60,188}
\pagecolor{bg}
\color{fg}

\begin{document}
\centering\Huge
\textbf{Energie} 
\[ 
        E=\left[kg\cdot m^2\cdot s ^{-2}=N\cdot m=W\cdot s=J\right]=\sum_{i}^{}E_i\quad E_{\text{kin}}=\dfrac{1}{2}mv^2\quad E_{\text{rot}}=\dfrac{1}{2}I\omega ^2\quad E_{\text{pot}}=mgh\quad E=mc ^2
\] 
ist die Größe, die die abgegebene Strahlung oder verrichtete Arbeit beschreibt. In einem System bleibt sie immer erhalten.
\end{document}
