\documentclass{article}

\usepackage[paperwidth=1920px,paperheight=1080px]{geometry}
\usepackage[T1]{fontenc}
\usepackage{amsmath,amssymb}
\usepackage[ISO]{diffcoeff}
\usepackage{xcolor}
\usepackage{color}

\definecolor{bg}{RGB}{38,38,38}
\definecolor{fg}{RGB}{100,100,100}
\pagecolor{bg}
\color{fg}

\begin{document}
\Huge
\textit{The Murders in the Rue Morgue}\\\\
What song the Syrens sang, or what name Achilles assumed when he hid himself among women, although puzzling questions, are not beyond \textit{all} conjecture.\\\indent
The mental features discoursed of as the analytical, are, in themselves, but little susceptible of analysis. 
We appreciate them only in their effects. 
We know of them, among other things, that they are always to their possessor, when inordinately possessed, a source of the liveliest enjoyment.
As the strong man exults in his physical ability, delighting in such exercises as call his muscles into action, so glories the analyst in that moral activity which \textit{disentangles}.
He derives pleasure from even the most trivial occupations bringing his talent into play.
He is fond of enigmas, of conundrums, of hieroglyphics; exhibiting in his solutions of each a degree of acumen which appears to the ordinary apprehension preternatural.
His results, brought about by the very soul and essence of method, have, in truth, the whole air of intuition.\\\indent
The faculty of resolution is possibly much invigorated by mathematical study, and especially by that highest branch of it which, unjustly, and merely on account of its retrograde operations, has been called, as if \textit{par excellence}, analysis. 
Yet to calculate is not in itself to analyse.\\
\null\hfill -- \textsc{Edgar Allan Poe}
\end{document}
